\section{Introduction}
\pagenumbering{arabic}
\renewcommand\frelo{\S\thesubsection}
\setcounter{page}{4}
\np The Dats language is a text-based programming language use for music composition.
It intents to promote the readability of ASCII music sheets and the compilation of these into
 a soundfile.
This language is fairly new and still in experimental so other features such trills, tie,
repeat, glissando, are not implemented but are considered to be in the future. It preserves the
concepts used in engraving music sheets via a syntatically similar one. This is
recognized by the use of similar terminologies like:

\begin{multicols}{2}
\begin{itemize}
\renewcommand{\labelitemi}{--}
\item staff
\item repeat
\item note
\item rest
\item octave
\item bpm (beats per minute)
\item semitone
\end{itemize}
\end{multicols}

\np Polyphony in
