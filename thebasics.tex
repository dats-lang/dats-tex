\section{The basics}

\np Toward playing music, there must be series of musical notes written for the player to read.
For dats, a musical note is declared with the keyword \verb+n+. \verb+n+ must contain two arguments,
a length and a note, each are separated by a comma:


\begin{Verbatim}[frame=single]
       n <length>, <note>;
\end{Verbatim}

%\begin{center}
%\includegraphics{s1.pdf}
%\end{center}

\np an example of such declaration is:
\begin{Verbatim}[frame=single]
       n 4, c4;
\end{Verbatim}

This is a declaration of musical note of length 1/4, playing the note c4. The length
is just the denominator of how much measure the musical note took.

\np Such musical notes, (and musical rests), are always declared inside a \textit{staff block}:

\begin{Verbatim}[frame=single]
      staff foo {
        n 4, c4;
      }
\end{Verbatim}

\begin{center}
\includegraphics{notes/c4}
\end{center}

\np Now, copy this to a file named t.dats and ran it with, `dats t.dats`... and you
see no sound file.

\begin{Verbatim}[frame=single]
      $ dats t.dats
      Symbol table of t.dats
        IDENTIFIER              TYPE
      
        foo                     staff
      $ ls
      $
\end{Verbatim}

this is because dats only read the staff and that's it; it didn't perform the synthesis of the
sound of each note. Such operation is performed inside the \textit{main block}. This main block
is declared with the \textgray{main} keyword.

\np First, we process the staff using a synthesizer. The output of this synthesizer
is then stored into a variable, usually a type of \textgray{track}.

\begin{Verbatim}[frame=single]
      staff foo {
        n 4, c4;
      }

      main {
        track bar = synth.kpa(foo);
        write("w.wav", bar);
      }
\end{Verbatim}

\np This is then written into a file named, "w.wav". (Currently, Dats only supports writing wav files.)

\np To append (or concatenate) one or more track to another track, you may simply just add a comma, followed by the track
you were appending with:

\begin{Verbatim}[frame=single]
      track tr1 = synth.kpa(/* some staff */);
      track tr2 = synth.kpa(/* some staff */);
      track tr3 = tr1, tr2, tr2;
\end{Verbatim}

\subsection{Track types}

\np There exists two types of \textgray{track}; mono and stereo. A mono track signifies a track
of one (1) channel and a stereo track signifies a track of two (2) channels.
A track of this type is specified with the \textgray{mono} and \textgray{stereo}
\textit{track type specifier}. By default, a track declared as stereo.

\np 

\subsubsection{Mono track}
\subsubsection{Stereo track}





